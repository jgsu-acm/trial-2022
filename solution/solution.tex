\documentclass{../cpct/ctsol}

\title{ACM 算法实验室 22 届成员选拔赛}
\date{2022 年 10 月 16 日}

\begin{document}

\maketitle
\addsolution{Zxilly 的饼}{Algor}{模拟}
\addsolution{置换反应}{AgOH}{模拟}
\addsolution{集和}{AgOH}{思维}
\addsolution{购物}{GlenBzc}{贪心,前缀和}
\addsolution{队形摆好}{Zxilly}{贪心}
\addsolution{最近等值对}{Algor}{贪心,\texttt{map}}
\addsolution{井字棋}{Algor}{模拟}
\addsolution{表演型卷王}{Tifa}{区间并}
\addsolution{跃迁}{AgOH}{期望 DP}

\section*{题目概览}

\solutiontab

\section*{鸣谢}

感谢 \href{https://github.com/Tiphereth-A}{\@Tifa} 大佬参与本次比赛的出题工作。

\makesolution
\section*{做法}

根据题意模拟即可。

\section*{标程}

\std{A}

\makesolution
\section*{做法}

根据题意模拟即可。

\section*{标程}

\std{B}

\makesolution
\section*{做法}

易发现 $A+A = \{2,3,\cdots,2n \}$,故 $|A+A|=2n-1$。

\section*{标程}

\std{C}

\makesolution
\section*{做法}

显然我们每次选择最贵的 $x_i$ 件购买即可使得这 $x_i$ 件中最便宜的 $y_i$ 总和最大,故我们首先对 $a_i$ 进行升序排序。

又因为有多次询问,每次询问相当于询问 $[n-x+1, n-x+y]$ 的区间和,故使用前缀和优化即可,时间复杂度 $O(n+q)$。

\section*{标程}

\std{D}

\makesolution
\section*{做法}

显然每班应达到的长度为各班人数的平均值,若总人数不能整除班级数则一定不能达成每班人数相等的目标。

考虑最靠左的人数不等于平均值的班级,若其人数小于平均值,我们显然会以一个人数大于平均值的班级来填充它(不论是否能填满),那么用哪个人数大于平均值的班级来填充呢?很显然要使用右边离它最近的那个(因为这样 $k$ 最小)。

考虑最靠左的人数不等于平均值的班级,若其人数大于平均值,我们显然会用它去填充一个人数小于平均值的班级(不论是否能填满),同理还是要使用右边离它最近的那个。

故只需从左到右扫一遍,过程中使用两个栈分别维护人数大于/小于平均值的班级。若发现一个人数小于/大于平均值的班级,即尽量在相应的栈中弹出人数大于/小于平均值的班级,并维护答案即可。

\section*{标程}

\std{E}

\makesolution
\section*{做法}

注意到值域很大($1 \leq a_i \leq {10}^9$),故我们可以使用一个 \lstinline{map} 或 \lstinline{unordered_map} 维护每个出现过的值之前出现过的位置。

一个数字 $a_i$ 若出现三次,出现位置分别为 $a,b,c$,之前的 $q-p$ 最小值为 $b-a$,我们只需要判断 $c-b$ 是否能更新最小值即可,而不需要判断 $c-a$(因为 $c-b$ 一定不会比 $c-a$ 更大),故对于每个值我们只需维护其上一次出现的位置即足够。

\section*{标程}

\std{F}

\makesolution
\section*{做法}

\verb|cout|

\section*{标程}

\std{G}

\makesolution
\section*{做法}

\verb|cout|

\section*{标程}

\std{H}

\makesolution
\section*{做法}

经典的随机游走模型,采用期望 DP 处理:

\begin{itemize}
    \item 状态设计:$dp[i]$ 代表从 $i$ 跃迁到 $i+1$ 期望花费的时长
    \item 初始状态:$dp[1] = 1$
    \item 转移方程:$$dp[i] = \cfrac{(1+dp[i-1]+1)+1}{2}$$
    \item 最终结果:$$\sum\limits_{i=1}^{n-1} dp[i]$$
\end{itemize}

转移方程中分子前半部分代表先走回 $i-1$ 再走到 $i$ 然后再走到 $i+1$,后半部分代表直接走到 $i+1$。

可以模拟分数,也可以推出 $dp$ 的通项公式及前 $n$ 项和公式直接算出结果:

通项公式:$dp[i] = 3-2^{2-i}$

前 $n$ 项和公式:$3n+2^{2-i}-4$

\section*{标程}

\std{I}

\end{document}
